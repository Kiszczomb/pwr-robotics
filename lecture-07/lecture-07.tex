\documentclass[12pt, a4paper]{extarticle}
\usepackage[a4paper, margin=1.3cm]{geometry}
\usepackage[T1]{fontenc} % For polish characters in author name
\usepackage[utf8]{inputenc} % For global UTF-8 encoding
\usepackage{booktabs} % For "\specialrule" command
\usepackage{graphicx} % For graphics
\graphicspath{ {./images/} }
\usepackage{subfig} % For subfigures (images side by side)
\usepackage{amsmath} % For math
\usepackage{icomma} % For "intelligent" comma
\usepackage{multirow} % For generated table
\usepackage{float} % For table positioning
\usepackage{gensymb} % For degree symbol
%\usepackage{hyperref} % For hypertext links
\usepackage[bottom]{footmisc} % For footnotes on bottom of page
\usepackage[table,xcdraw]{xcolor} % For colored tables
\usepackage{multicol} % For multicolum lists
\usepackage[pdfusetitle]{hyperref} % For automatic PDF metadata
\usepackage[makeroom]{cancel} % For rossing out / canceling in math
\usepackage{pdflscape} % For landscape PDF
\usepackage{listings} % For source code listings (snippets)
\usepackage{svg} % For SVG graphics
\usepackage[super]{nth} % for "1st", "2nd" and "i-th" numbering: \nth{2}
\usepackage{arydshln} % for dashed line in matrix
\usepackage{tikz}
\usetikzlibrary{patterns}

\usepackage{xcolor}
\hypersetup{
	colorlinks,
	linkcolor={green!50!black},
	citecolor={blue!50!black},
	urlcolor={blue!80!black}
}



%\usepackage[pdftex,
%pdfauthor={Kacper Jastrzębski, 260607@student.pwr.edu.pl},
%pdftitle={The Title},
%pdfsubject={The Subject},
%pdfkeywords={Some Keywords},
%pdfproducer={Latex with hyperref, or other system},
%pdfcreator={pdflatex, or other tool}]{hyperref} % For manual PDF metadata

\usepackage{letltxmacro} % For new style roots with "closing"
\makeatletter
\let\oldr@@t\r@@t
\def\r@@t#1#2{%
	\setbox0=\hbox{$\oldr@@t#1{#2\,}$}\dimen0=\ht0
	\advance\dimen0-0.2\ht0
	\setbox2=\hbox{\vrule height\ht0 depth -\dimen0}%
	{\box0\lower0.4pt\box2}}
\LetLtxMacro{\oldsqrt}{\sqrt}
\renewcommand*{\sqrt}[2][\ ]{\oldsqrt[#1]{#2} }
\makeatother

\renewcommand{\baselinestretch}{1.0}
%\renewcommand{\familydefault}{\sfdefault}
%\setlength{\parindent}{4em}
\setlength{\parskip}{0em}

\setlength{\arrayrulewidth}{0.2mm}
\setlength{\tabcolsep}{12pt}
\renewcommand{\arraystretch}{1.2}

\newcommand{\experiment}{E105}

% setting the line height in gather env.
\setlength{\jot}{3mm}


\title{
	Introduction to Robotics \\
	\vspace{\baselineskip}
	\large
	\textbf{Lecture 7}
}
\author{
	Kacper Jastrzębski\\
	260607@student.pwr.edu.pl
}
\date{Date: Tuesday 11:15, 18-04-2023}

\urldef\urlwiki\url{https://en.wikipedia.org/wiki/Denavit%E2%80%93Hartenberg_parameters}


\begin{document}

	\maketitle
	\vspace{1.5cm}

	\tableofcontents

	\pagebreak

	\section{Rewind –– Inverse kinematics}

	Inverse kinematics -- we have given position and orientation and we have to find the configuration that realizes this problem. One of the methods, jacobi based, for solving that kind of task.\\

	Given: $x = k(1), x_f \in X$

	\begin{align}\label{key}
		q_{i+1} &= q_i + \xi_i J^\#(q_i)(x_f - k(q_i))\\
		J^\# &= J^T (JJ^T)^{-1}
	\end{align}

	This is an iterative algorithm, so $q_0 \rightarrow q_1 \rightarrow q$. That kind of thask comes in two flavours:
	\begin{itemize}
		\item regular configuration: $rank\; J(q) = m$
		\item singular configuration: $rank\; J(q) < m$
	\end{itemize}

	\subsection{Singular case}

	How to determine singular configuratoin:

	\begin{enumerate}
		\item $rank\; J(q)$\\
		\textit{J as $m \times n$ rectangle}
		select all $m \times m$ -- sub-matrices of J (there are $\binom{n}{k} = \dfrac{n!}{m! (n-m)!}$)
		singular cont: if all determinants of all submatrices are equal to 0
		\item $det\left(J(q) J^T(q)\right)$\\
		Example calculation \textit{(3 DOF planar pendulum)}
		\begin{multicols}{2}
				\begin{align}
				\begin{bmatrix}
					x\\
					y
				\end{bmatrix} =
				\begin{bmatrix}
					k_1(q)\\
					k_2(q)
				\end{bmatrix} =
				\begin{bmatrix}
					a_1 c_1 + a_2 c_{12} + a_3 c_{123} \\
					a_1 s_1 + a_2 s_{12} + a_3 s_{123}
				\end{bmatrix}
			\end{align}
			\\
			\begin{align}
				q &=
				\begin{bmatrix}
					q_1 \\ q_2 \\ q_3
				\end{bmatrix} \\
				J &= \dfrac{\partial k}{\partial q} \\
				m &= dim\; X \\
				n &= dim\; Q
			\end{align}
		\end{multicols}
		\begin{align}
			det\;J_{23} &= a_1 a_2 s_3 = 0\\
			det\;J_{12} &= a_1 a_2 s_@ + a_1 a_3 s_{23} = 0\\
			det\;J_{13} &= a_1 a_3 s_{23} + a_2 a_3 s_3 = 0
		\end{align}
		therefore:
		\begin{align}
			s_3 &= 0 \rightarrow q_3 = \left\{0, \pi\right\} \\
			s_{23} &= 0 \rightarrow q_2 = \left\{0, \pi\right\} \\
			q &= \begin{bmatrix}
				* \\
				\left\{0, \pi\right\} \\
				\left\{0, \pi\right\}
			\end{bmatrix}
		\end{align}
		At regular configuration we can move in every direction infinitesimally (like arm in L shape). in singular it is not possible	(like straight extended arm).\\


	\end{enumerate}

	\section{Other methods}
	\subsection{Geometrical method}

	Works only or special kind of robots, lets excercise with planar double pendulum:\\

	\textit{photo}\\
	\begin{align}
		\begin{bmatrix}
			x \\ y
		\end{bmatrix} =
		\begin{bmatrix}
			a_1 c_1 + a_2 c_{12} \\ a_1 s_1 + a_2 S_{12}
		\end{bmatrix}
	\end{align}
	Having given $x_f$ and $y_f$ we can easily obtain $r = \sqrt{x_f^2 + y_f^2}$. Using cosinus theorem:
	\begin{align}
		r^2 &= a_1^2 + a_2^2 - 2 a_1 a_2 \cos \beta \rightarrow \beta \\
		a_2^2 &= a_1^2 + r^2 - 2 a_1 r \cos \alpha \rightarrow \alpha
	\end{align}
	therefore
	\begin{align}
		q_2 &= \beta - \pi \\
		q_1 &= \gamma + \alpha \\
		\tan \gamma &= \dfrac{x_f}{y_f} \rightarrow \gamma
	\end{align}
	There is also another solution, mirrored, called elbow-up or elbow-down.
	For singular case (extended arm)

	\subsection{Kinematics decoupling}
	Applicable only when three last axes cross at a single point.
	Let's have a manipulator with $n = 6$
	\begin{align}
		K(q) &=
		\left[
		\begin{array}{c;{2pt/2pt}c}
			R_0^6(q)  & d_0^6(q) \\ \hdashline[2pt/2pt]
			0 & 1
		\end{array}
		\right]\\
		r(q_1, q_2, q_3) &= d_0^6 - d_6 R_0^6 e_3 \\
		e_3 = \begin{bmatrix}
			0 \\ 0 \\ 1
		\end{bmatrix}
	\end{align}
	We used to calculate in forward kinematics this general approach:
	\begin{align}
		A_0^n(q) = \dots
	\end{align}
	Now we have case that:
	\begin{align}
		R_0^6 = R_0^3(q_1, q_2, q_3) R_0^6(q_4, q_5, q_6)
	\end{align}
	So at the end we decoupled this 6-dimensional problem in two 3-dimensional problems. This is always a good idea, because of the complexity
\end{document}
