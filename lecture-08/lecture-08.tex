\documentclass[12pt, a4paper]{extarticle}
\usepackage[a4paper, margin=1.3cm]{geometry}
\usepackage[T1]{fontenc} % For polish characters in author name
\usepackage[utf8]{inputenc} % For global UTF-8 encoding
\usepackage{booktabs} % For "\specialrule" command
\usepackage{graphicx} % For graphics
\graphicspath{ {./images/} }
\usepackage{subfig} % For subfigures (images side by side)
\usepackage{amsmath} % For math
\usepackage{icomma} % For "intelligent" comma
\usepackage{multirow} % For generated table
\usepackage{float} % For table positioning
\usepackage{gensymb} % For degree symbol
%\usepackage{hyperref} % For hypertext links
\usepackage[bottom]{footmisc} % For footnotes on bottom of page
\usepackage[table,xcdraw]{xcolor} % For colored tables
\usepackage{multicol} % For multicolum lists
\usepackage[pdfusetitle]{hyperref} % For automatic PDF metadata
\usepackage[makeroom]{cancel} % For rossing out / canceling in math
\usepackage{pdflscape} % For landscape PDF
\usepackage{listings} % For source code listings (snippets)
\usepackage{svg} % For SVG graphics
\usepackage[super]{nth} % for "1st", "2nd" and "i-th" numbering: \nth{2}
\usepackage{arydshln} % for dashed line in matrix
\usepackage{tikz}
\usetikzlibrary{patterns}

\usepackage{xcolor}
\hypersetup{
	colorlinks,
	linkcolor={green!50!black},
	citecolor={blue!50!black},
	urlcolor={blue!80!black}
}



%\usepackage[pdftex,
%pdfauthor={Kacper Jastrzębski, 260607@student.pwr.edu.pl},
%pdftitle={The Title},
%pdfsubject={The Subject},
%pdfkeywords={Some Keywords},
%pdfproducer={Latex with hyperref, or other system},
%pdfcreator={pdflatex, or other tool}]{hyperref} % For manual PDF metadata

\usepackage{letltxmacro} % For new style roots with "closing"
\makeatletter
\let\oldr@@t\r@@t
\def\r@@t#1#2{%
	\setbox0=\hbox{$\oldr@@t#1{#2\,}$}\dimen0=\ht0
	\advance\dimen0-0.2\ht0
	\setbox2=\hbox{\vrule height\ht0 depth -\dimen0}%
	{\box0\lower0.4pt\box2}}
\LetLtxMacro{\oldsqrt}{\sqrt}
\renewcommand*{\sqrt}[2][\ ]{\oldsqrt[#1]{#2} }
\makeatother

\renewcommand{\baselinestretch}{1.0}
%\renewcommand{\familydefault}{\sfdefault}
%\setlength{\parindent}{4em}
\setlength{\parskip}{0em}

\setlength{\arrayrulewidth}{0.2mm}
\setlength{\tabcolsep}{12pt}
\renewcommand{\arraystretch}{1.2}

\newcommand{\experiment}{E105}

% setting the line height in gather env.
\setlength{\jot}{3mm}


\title{
	Introduction to Robotics \\
	\vspace{\baselineskip}
	\large
	\textbf{Lecture 8}
}
\author{
	Kacper Jastrzębski\\
	260607@student.pwr.edu.pl
}
\date{Date: Tuesday 11:15, 25-04-2023}

\urldef\urlwiki\url{https://en.wikipedia.org/wiki/Denavit%E2%80%93Hartenberg_parameters}


\begin{document}

	\maketitle
	\vspace{1.5cm}

	\tableofcontents

	\pagebreak

	Until now we mainly considered manipulator, the robots we can encounter eg on factory floors. Today we'll talk about a different kind: mobile robots. As for manipulator, our goal is to find a model, mathematical description.

	\section{Mobile manipulators and how to model them}
	\begin{figure}[H]
		\centering
		\begin{tikzpicture}
			\draw[->] (0, 0) -- (5, 0) node[right] {$x$};
			\draw[->] (0, 0) -- (0, 5) node[above] at (1.9,3) {$f(x) \longrightarrow \text{TODO}$};
			\draw[blue] (2,2.25) -- (2,0);
			\draw[blue] (5,2) -- (5,0);
			\node at (3.5,1.1) {$P(a \le x \le b)$};
			\node at (0,-0.8) {$-\infty$};
			\node at (2,-0.4) {$a$};
			\node at (5,-0.4) {$b$};
		\end{tikzpicture}
		\caption{Model od unicycle (top view)}
	\end{figure}


	Let's consider the simplest mobile robot -- a unicycle. What is its confiquration (set of data required to uniquely describe the manipulator)

	\begin{align}
		q &= \begin{bmatrix}
			x \\
			y \\
			\theta
		\end{bmatrix}
	\end{align}

	With assumption that there is no lateral slippage:

	\begin{align}
		\dfrac{\partial y}{\partial x} &= \tan \theta = \dfrac{\sin \theta}{\cos \theta} \\
		\sin \theta \cdot \partial x - \cos \theta \cdot \partial y &= 0 \; \left/ \cdot \dfrac{\partial}{\partial t} \right. \\
		\sin \theta \cdot \dot{x} - \cos \theta \cdot \dot{y} &= 0 \\
		\begin{bmatrix}
			\sin \theta & -\cos \theta & 0
		\end{bmatrix}
		\begin{bmatrix}
			\dot{x} \\
			\dot{y} \\
			\dot{\theta}
		\end{bmatrix} &= \begin{bmatrix}
		0
	\end{bmatrix}
	\end{align}

	In robotics those constrians are called constraints in \textbf{Pfaff form}:

	\begin{align}
		A(q)\dot{q} &= 0 \\
		rank \; A(q) &= r
	\end{align}

	no longitudinal slippage, odometry, motion of point\\
	no lateral slppage and no longitudinal slippage coupled together \\
	constriants (2) x
	\begin{align}
		\begin{bmatrix}
			\sin \theta & -\cos \theta & 0 & 0 \\
			\cos \theta & \sin \theta & 0 & -R
		\end{bmatrix}
		\begin{bmatrix}
			\dot{x} \\
			\dot{y} \\
			\dot{\theta} \\
			\dot{\phi}
		\end{bmatrix}
	\end{align}

	Lets complicate the system:

	\section{Configuration vs controllability}

	n = 3 <- dimension of configuration space\\
	r = 1 <- number of constrians \\
	m = n - r = 2 <- difference (controls?)

	\subsection{Vector field vs vectors}

	Lets find a perpendicular vector field
	\begin{align}
		a_1 \sin \theta - a_2 \cos \theta + a_3 \cdot 0= 0 \;\forall\; q
	\end{align}

	\subsection{How to determine if system is controllable?}



\end{document}
