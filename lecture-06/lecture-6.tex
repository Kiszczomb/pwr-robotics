\documentclass[12pt, a4paper]{extarticle}
\usepackage[a4paper, margin=1.3cm]{geometry}
\usepackage[T1]{fontenc} % For polish characters in author name
\usepackage[utf8]{inputenc} % For global UTF-8 encoding
\usepackage{booktabs} % For "\specialrule" command
\usepackage{graphicx} % For graphics
\graphicspath{ {./images/} }
\usepackage{subfig} % For subfigures (images side by side)
\usepackage{amsmath} % For math
\usepackage{icomma} % For "intelligent" comma
\usepackage{multirow} % For generated table
\usepackage{float} % For table positioning
\usepackage{gensymb} % For degree symbol
%\usepackage{hyperref} % For hypertext links
\usepackage[bottom]{footmisc} % For footnotes on bottom of page
\usepackage[table,xcdraw]{xcolor} % For colored tables
\usepackage{multicol} % For multicolum lists
\usepackage[pdfusetitle]{hyperref} % For automatic PDF metadata
\usepackage[makeroom]{cancel} % For rossing out / canceling in math
\usepackage{pdflscape} % For landscape PDF
\usepackage{listings} % For source code listings (snippets)
\usepackage{svg} % For SVG graphics
\usepackage[super]{nth} % for "1st", "2nd" and "i-th" numbering: \nth{2}
\usepackage{arydshln} % for dashed line in matrix
\usepackage{tikz}
\usetikzlibrary{patterns}

\usepackage{xcolor}
\hypersetup{
	colorlinks,
	linkcolor={green!50!black},
	citecolor={blue!50!black},
	urlcolor={blue!80!black}
}



%\usepackage[pdftex,
%pdfauthor={Kacper Jastrzębski, 260607@student.pwr.edu.pl},
%pdftitle={The Title},
%pdfsubject={The Subject},
%pdfkeywords={Some Keywords},
%pdfproducer={Latex with hyperref, or other system},
%pdfcreator={pdflatex, or other tool}]{hyperref} % For manual PDF metadata

\usepackage{letltxmacro} % For new style roots with "closing"
\makeatletter
\let\oldr@@t\r@@t
\def\r@@t#1#2{%
	\setbox0=\hbox{$\oldr@@t#1{#2\,}$}\dimen0=\ht0
	\advance\dimen0-0.2\ht0
	\setbox2=\hbox{\vrule height\ht0 depth -\dimen0}%
	{\box0\lower0.4pt\box2}}
\LetLtxMacro{\oldsqrt}{\sqrt}
\renewcommand*{\sqrt}[2][\ ]{\oldsqrt[#1]{#2} }
\makeatother

\renewcommand{\baselinestretch}{1.0}
%\renewcommand{\familydefault}{\sfdefault}
%\setlength{\parindent}{4em}
\setlength{\parskip}{0em}

\setlength{\arrayrulewidth}{0.2mm}
\setlength{\tabcolsep}{12pt}
\renewcommand{\arraystretch}{1.2}

\newcommand{\experiment}{E105}

% setting the line height in gather env.
\setlength{\jot}{3mm}


\title{
	Introduction to Robotics \\
	\vspace{\baselineskip}
	\large
	\textbf{Lecture 6}
}
\author{
	Kacper Jastrzębski\\
	260607@student.pwr.edu.pl
}
\date{Date: Tuesday 11:15, 04-04-2023}

\urldef\urlwiki\url{https://en.wikipedia.org/wiki/Denavit%E2%80%93Hartenberg_parameters}


\begin{document}

	\maketitle
	\vspace{1.5cm}

	\tableofcontents

	\pagebreak

	\section{Forward kinematics}
	\textit{From Wikipedia:} Forward kinematics refers to the use of the kinematic equations of a robot to compute the position of the end-effector from specified values for the joint parameters (configuration vectors)\\

	\begin{minipage}{\textwidth}
		\begin{equation}\label{key}
			Q\footnote{Configuration space} \ni q \rightarrow x\footnote{Position an orientation of end-effector} = k(q) \quad \text{where} \quad x \in X\footnote{Task space}
		\end{equation}
	\end{minipage}

	\section{Different coordinates systems}

	\subsection{Cartesian}
	Unique -- point $\begin{bmatrix}
		0 & 0 & 0
	\end{bmatrix}$ describes one point exactly.

	\subsection{Cylindrical}
	Non-unique -- the same point can be described by $r=0$ and any angle $\gamma \in [0, 2\pi]$.
	Sensitivity to error -- for

	\subsection{Spherical}

	\section{Inverse kinematics}
	Having given point (generalized position) find the configuration that realizes it.\\
	Given the task space and forward kinematics find such configuration(s) that solve the equation.\\
	Given: $x_p \in X, k=k(q)$ find: $q^*: \; x_f=k(q^*) $.

	\begin{minipage}{\textwidth}
		\begin{align}
			x^{(t)} &= k(q^{(t)}) \quad \left/ \dfrac{\partial}{\partial t} \right. \\
			\dot{x} &= \dfrac{\partial k(q)}{\partial q} \cdot \dot{q} = J(q) \dot{q}
		\end{align}
	\end{minipage}

	\subsection{Double pendulum}
	Photo

	\subsection{Custom example}
	Photo\\

	Function approximation using Taylor series\\

	How to get $q_{i+1}$ from the complicated equation:\\

	\begin{align}
		\xi \left(x_f - k(q_i)\right) &= J(q_i)\left(q_{i+1} - q_i\right) \\
		q_{i+1} &= q_i + \xi J^{-1}(q_i)(x_f - k(q_i))
	\end{align}

	Newton algorithm of inverse kinematics for non-redundant manipulator.\\

	Only for inversible Jacobi matrix (aka. entering the singular configuration)\\

	\textbf{Stop condition} -- when the difference between $q_i$ and $q_f$ is small enough.\\

	regular configuration $q$: $rank J(q) = m$ (full rank)\\

	singular configuration $q$: $rank J(q) < m$\\

	Checking the rank: 2x3 -> at least one determinant of square submatrix must be non-zero\\

	What about redundant manipulators, with non-square matrices: \\

	pseudo-inverse of Jacobi matrix: (still not working for singular configuration)
	\begin{equation}\label{key}
		J^\# = J^T(J J^T)^{-1}
	\end{equation}
	Manipulabity matrix
	\begin{equation}\label{key}
		J_{(q)} J^T_{(q)} = M(q)
	\end{equation}

	3rd method -- robust (against singularity) inverse:
	\begin{equation}\label{key}
		J^\# = J^T(J J^T + \lambda I_m)
	\end{equation}

	When to switch to robust? When the determinant of manipulabity matrix is below given threshold

\end{document}
