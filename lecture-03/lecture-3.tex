\documentclass[12pt, a4paper]{extarticle}
\usepackage[a4paper, margin=1.3cm]{geometry}
\usepackage[T1]{fontenc} % For polish characters in author name
\usepackage[utf8]{inputenc} % For global UTF-8 encoding
\usepackage{booktabs} % For "\specialrule" command
\usepackage{graphicx} % For graphics
\graphicspath{ {./images/} }
\usepackage{subfig} % For subfigures (images side by side)
\usepackage{amsmath} % For math
\usepackage{icomma} % For "intelligent" comma
\usepackage{multirow} % For generated table
\usepackage{float} % For table positioning
\usepackage{gensymb} % For degree symbol
%\usepackage{hyperref} % For hypertext links
\usepackage[bottom]{footmisc} % For footnotes on bottom of page
\usepackage[table,xcdraw]{xcolor} % For colored tables
\usepackage{multicol} % For multicolum lists
\usepackage[pdfusetitle]{hyperref} % For automatic PDF metadata
\usepackage[makeroom]{cancel} % For rossing out / canceling in math
\usepackage{pdflscape} % For landscape PDF
\usepackage{listings} % For source code listings (snippets)
\usepackage{svg} % For SVG graphics
\usepackage[super]{nth} % for "1st", "2nd" and "i-th" numbering: \nth{2}
\usepackage{arydshln} % for dashed line in matrix
\usepackage{bm} % for bold math mode, using command:  \bm{g}


\usepackage{xcolor}
\hypersetup{
	colorlinks,
	linkcolor={green!50!black},
	citecolor={blue!50!black},
	urlcolor={blue!80!black}
}



%\usepackage[pdftex,
%pdfauthor={Kacper Jastrzębski, 260607@student.pwr.edu.pl},
%pdftitle={The Title},
%pdfsubject={The Subject},
%pdfkeywords={Some Keywords},
%pdfproducer={Latex with hyperref, or other system},
%pdfcreator={pdflatex, or other tool}]{hyperref} % For manual PDF metadata

\usepackage{letltxmacro} % For new style roots with "closing"
\makeatletter
\let\oldr@@t\r@@t
\def\r@@t#1#2{%
	\setbox0=\hbox{$\oldr@@t#1{#2\,}$}\dimen0=\ht0
	\advance\dimen0-0.2\ht0
	\setbox2=\hbox{\vrule height\ht0 depth -\dimen0}%
	{\box0\lower0.4pt\box2}}
\LetLtxMacro{\oldsqrt}{\sqrt}
\renewcommand*{\sqrt}[2][\ ]{\oldsqrt[#1]{#2} }
\makeatother

\renewcommand{\baselinestretch}{1.0}
%\renewcommand{\familydefault}{\sfdefault}
%\setlength{\parindent}{4em}
\setlength{\parskip}{0em}

\setlength{\arrayrulewidth}{0.2mm}
\setlength{\tabcolsep}{12pt}
\renewcommand{\arraystretch}{1.2}

\newcommand{\experiment}{E105}

% setting the line height in gather env.
\setlength{\jot}{3mm}


\title{
	Introduction to Robotics \\
	\vspace{\baselineskip}
	\large
	\textbf{Lecture 3}
}
\author{
	Kacper Jastrzębski\\
	260607@student.pwr.edu.pl
}
\date{Date: Tuesday 11:15, 14-03-2023}

%\urldef\urlwiki\url{https://en.wikipedia.org/wiki/Denavit%E2%80%93Hartenberg_parameters}


\begin{document}

	\maketitle
	\vspace{1.5cm}

	\tableofcontents

	\pagebreak

	\section{Block Matrix of translation and rotation}

	To create single object consisting of rotational and translation matrices we can group them in so called \textit{block matrix (A)}. For easier calculations, we can make it square by filling it with constant values that does not interfere with operations performed on it (zeros and ones in our case). The final object will look something like this:

	\begin{minipage}{\textwidth}
		\begin{equation}\label{key}
			A^1_0 = \begin{bmatrix}
				R^1_0 & d^1_0 \\
				\bm{0}\footnote{this is actually a $3\times 1$ matrix, so it should be denoted $ [\;0\;0\;0\;]$, but for the simplification we'll just be using \textbf{0}} & \bm{1}\footnote{Similarly, in this case \textbf{1} represents a $1 \times 1$ unit matrix. }
			\end{bmatrix}
		\end{equation}
	\end{minipage}
	\\
	\\
	Let's check if this object obeys the \textit{chain rule}:

	\begin{equation}\label{key}
		A^2_0 = \begin{bmatrix}
			R^2_0 & d^2_0 \\
			0 & 1
		\end{bmatrix} =
		\begin{bmatrix}
			R^1_0 & d^1_0 \\ 0 & 1
		\end{bmatrix}
		\begin{bmatrix}
			R^2_1 & d^2_1 \\ 0 & 1
		\end{bmatrix} =
		\left[
		\begin{array}{c;{2pt/2pt}c}
			R^1_0 R^2_1 & R^1_0 d^2_1 + d^1_0 \\ \hdashline[2pt/2pt]
			0 & 1
		\end{array}
		\right]
	\end{equation}

	As we can see, the chain multiplication can be performed without issues. But what about an operation like $p_0 = A^1_0 p_1$? We cannot perform it, because sizes of vector $p$ and object $A$ are different. Let's add one missing dimension to the vector then:

	\begin{equation}\label{key}
		\begin{bmatrix}
			p_0 \\ 1
		\end{bmatrix} =
		A^1_0 \cdot
		\begin{bmatrix}
			p_1 \\ 1
		\end{bmatrix}
	\end{equation}

	We can shortly denote is

	\section{Building blocks for SE(3) group}

	\subsection{Rotational functions}

	\subsection{Translation functions}

	\subsection{Order of operations}

	\subsection{How to check if matrix belongs to SE(3)?}

	\section{Euler angles}

	\subsection{Regular case}

	\subsection{Simple case}


\end{document}
