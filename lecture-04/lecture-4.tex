\documentclass[12pt, a4paper]{extarticle}
\usepackage[a4paper, margin=1.3cm]{geometry}
\usepackage[T1]{fontenc} % For polish characters in author name
\usepackage[utf8]{inputenc} % For global UTF-8 encoding
\usepackage{booktabs} % For "\specialrule" command
\usepackage{graphicx} % For graphics
\graphicspath{ {./images/} }
\usepackage{subfig} % For subfigures (images side by side)
\usepackage{amsmath} % For math
\usepackage{icomma} % For "intelligent" comma
\usepackage{multirow} % For generated table
\usepackage{float} % For table positioning
\usepackage{gensymb} % For degree symbol
%\usepackage{hyperref} % For hypertext links
\usepackage[bottom]{footmisc} % For footnotes on bottom of page
\usepackage[table,xcdraw]{xcolor} % For colored tables
\usepackage{multicol} % For multicolum lists
\usepackage[pdfusetitle]{hyperref} % For automatic PDF metadata
\usepackage[makeroom]{cancel} % For rossing out / canceling in math
\usepackage{pdflscape} % For landscape PDF
\usepackage{listings} % For source code listings (snippets)
\usepackage{svg} % For SVG graphics
\usepackage[super]{nth} % for "1st", "2nd" and "i-th" numbering: \nth{2}
\usepackage{arydshln} % for dashed line in matrix
\usepackage{tikz}
\usetikzlibrary{patterns}

\usepackage{xcolor}
\hypersetup{
	colorlinks,
	linkcolor={green!50!black},
	citecolor={blue!50!black},
	urlcolor={blue!80!black}
}



%\usepackage[pdftex,
%pdfauthor={Kacper Jastrzębski, 260607@student.pwr.edu.pl},
%pdftitle={The Title},
%pdfsubject={The Subject},
%pdfkeywords={Some Keywords},
%pdfproducer={Latex with hyperref, or other system},
%pdfcreator={pdflatex, or other tool}]{hyperref} % For manual PDF metadata

\usepackage{letltxmacro} % For new style roots with "closing"
\makeatletter
\let\oldr@@t\r@@t
\def\r@@t#1#2{%
	\setbox0=\hbox{$\oldr@@t#1{#2\,}$}\dimen0=\ht0
	\advance\dimen0-0.2\ht0
	\setbox2=\hbox{\vrule height\ht0 depth -\dimen0}%
	{\box0\lower0.4pt\box2}}
\LetLtxMacro{\oldsqrt}{\sqrt}
\renewcommand*{\sqrt}[2][\ ]{\oldsqrt[#1]{#2} }
\makeatother

\renewcommand{\baselinestretch}{1.0}
%\renewcommand{\familydefault}{\sfdefault}
%\setlength{\parindent}{4em}
\setlength{\parskip}{0em}

\setlength{\arrayrulewidth}{0.2mm}
\setlength{\tabcolsep}{12pt}
\renewcommand{\arraystretch}{1.2}

\newcommand{\experiment}{E105}

% setting the line height in gather env.
\setlength{\jot}{3mm}


\title{
	Introduction to Robotics \\
	\vspace{\baselineskip}
	\large
	\textbf{Lecture 4}
}
\author{
	Kacper Jastrzębski\\
	260607@student.pwr.edu.pl
}
\date{Date: Tuesday 11:15, 21-03-2023}

\urldef\urlwiki\url{https://en.wikipedia.org/wiki/Denavit%E2%80%93Hartenberg_parameters}


\begin{document}

	\maketitle
	\vspace{1.5cm}

	\tableofcontents

	\pagebreak

	\section{Forward kinematics}

	Let's recapitulate the naming conventions for a exemplary manipulator:

	% Note. This illustration was originally made with PSTricks. Conversion to
	% PGF/TikZ was straightforward. However, I could probably have made it more
	% elegant.

	% Define a variable as a length
	% Input:
	%   #1 Variable name
	%   #2 Value
	%
	% Example:
	%   \nvar{\varx}{2cm}
	\newcommand{\nvar}[2]{%
		\newlength{#1}
		\setlength{#1}{#2}
	}

	% Define a few constants for drawing
	\nvar{\dg}{0.3cm}
	\def\dw{0.25}\def\dh{0.5}
	\nvar{\ddx}{1.5cm}

	% Define commands for links, joints and such
	\def\link{\draw [double distance=1.5mm, very thick] (0,0)--}
	\def\joint{%
		\filldraw [fill=white] (0,0) circle (5pt);
		\fill[black] circle (2pt);
	}
	\def\grip{%
		\draw[ultra thick](0cm,\dg)--(0cm,-\dg);
		\fill (0cm, 0.5\dg)+(0cm,1.5pt) -- +(0.6\dg,0cm) -- +(0pt,-1.5pt);
		\fill (0cm, -0.5\dg)+(0cm,1.5pt) -- +(0.6\dg,0cm) -- +(0pt,-1.5pt);
	}
	\def\robotbase{%
		\draw[rounded corners=8pt] (-\dw,-\dh)-- (-\dw, 0) --
		(0,\dh)--(\dw,0)--(\dw,-\dh);
		\draw (-0.5,-\dh)-- (0.5,-\dh);
		\fill[pattern=north east lines] (-0.5,-1) rectangle (0.5,-\dh);
	}

	% Draw an angle annotation
	% Input:
	%   #1 Angle
	%   #2 Label
	% Example:
	%   \angann{30}{$\theta_1$}
	\newcommand{\angann}[2]{%
		\begin{scope}[red]
			\draw [dashed, red] (0,0) -- (1.2\ddx,0pt);
			\draw [->, shorten >=3.5pt] (\ddx,0pt) arc (0:#1:\ddx);
			% Unfortunately automatic node placement on an arc is not supported yet.
			% We therefore have to compute an appropriate coordinate ourselves.
			\node at (#1/2-2:\ddx+8pt) {#2};
		\end{scope}
	}

	% Draw line annotation
	% Input:
	%   #1 Line offset (optional)
	%   #2 Line angle
	%   #3 Line length
	%   #5 Line label
	% Example:
	%   \lineann[1]{30}{2}{$L_1$}
	\newcommand{\lineann}[4][0.5]{%
		\begin{scope}[rotate=#2, blue,inner sep=2pt]
			\draw[dashed, blue!40] (0,0) -- +(0,#1)
			node [coordinate, near end] (a) {};
			\draw[dashed, blue!40] (#3,0) -- +(0,#1)
			node [coordinate, near end] (b) {};
			\draw[|<->|] (a) -- node[fill=white] {#4} (b);
		\end{scope}
	}

	% Define the kinematic parameters of the three link manipulator.
	\def\thetaone{30}
	\def\Lone{2}
	\def\thetatwo{30}
	\def\Ltwo{2}
	\def\thetathree{30}
	\def\Lthree{1}

	\begin{center}
		\begin{tikzpicture}
			\robotbase
			\angann{\thetaone}{$\theta_1$}
			\lineann[0.7]{\thetaone}{\Lone}{$L_1$}
			\link(\thetaone:\Lone);
			\joint
			\begin{scope}[shift=(\thetaone:\Lone), rotate=\thetaone]
				\angann{\thetatwo}{$\theta_2$}
				\lineann[-1.5]{\thetatwo}{\Ltwo}{$L_2$}
				\link(\thetatwo:\Ltwo);
				\joint
				\begin{scope}[shift=(\thetatwo:\Ltwo), rotate=\thetatwo]
					\angann{\thetathree}{$\theta_3$}
					\lineann[0.7]{\thetathree}{\Lthree}{$L_3$}
					\draw [dashed, red,rotate=\thetathree] (0,0) -- (1.2\ddx,0pt);
					\link(\thetathree:\Lthree);
					\joint
					\begin{scope}[shift=(\thetathree:\Lthree), rotate=\thetathree]
						\grip
					\end{scope}
				\end{scope}
			\end{scope}
		\end{tikzpicture}
	\end{center}


	\section{Denavit-Hartenberg (1955) solution}

	Jacques Denavit and Richard Hartenberg introduced this convention in 1955 in order to standardize the coordinate frames for spatial linkages\footnote{Description borrowed from Wikipedia: \href{https://en.wikipedia.org/wiki/Denavit\%E2\%80\%93Hartenberg_parameters}{Denavit–Hartenberg parameters}.}. They came up with an universal algorithm for describing the motion (or in other words: attaching a reference frames to the links) of a manipulator.

	\subsection{Preliminary assumptions}
	\begin{enumerate}
		\setlength\itemsep{-0.3em}
		\item motion allowed only along z-axis
		\item rigid body assumed
	\end{enumerate}
	\noindent
	\subsection{Algorithm:}
	\begin{enumerate}
		\setlength\itemsep{-0.3em}
		\item Step: assign axes of rotation $z_0 \dots z_{n-1}$
		\item Step: describe base frame $O_0 x_0 y_0 z_0$\footnote{Axis should be chosen wisely, in respect to the surroundings, context, and the use case.}
		\item Step: create a loop $i=1, \dots, n-1$ (repeat steps 4-6)
		\item Step: determine $O_i$ (the origin of next frame), consider 3 cases:
		\begin{enumerate}
			\item case: $O_i = z_{i-1}  \bigcap z_i$
			\item case parallel: a point where normal line passing through $O_{i0-1}$ crosses $Z_i$
			\item case: a point where normal line to both $Z_{i-1}$ and $Z_i$ crosses $Z_i$
		\end{enumerate}
		\item Step: determine $x_i$ axis, for each case:
		\begin{enumerate}
			\item $x_i = z_{i-1} \times Z_i$
			\item b and c: $x_i$ along normal line selected previously
		\end{enumerate}
		\item Step: calculate missing axis $y_i$ such the $x_i y_i z_i$ is a right-handed frame
		\item Step: end-effector frame:
		\begin{enumerate}
			\item origin $O_n$ -- between fingers of a grabbing, two fingered effector
			\item $z_n \; || \; z_{n-1}$ -- inherited from the last joint
			\item $y_n$ -- finger motion direction
			\item $x_n \rightarrow x_n y_n z_n \rightarrow $ right-handed
		\end{enumerate}
		\item Step: determine D-H parameters described in table below:

		\begin{minipage}{\textwidth}
			\begin{table}[H]
				\centering
				\resizebox{8cm}{!}{
				\begin{tabular}{|c|c|c|c|c|}
					\hline
					& $\theta_i$ & $d_i$ & $a_i$ & $\alpha_i$\\
					\hline
					1 & $\theta_1$ & $d_1$ & $a_1$ & $\alpha_1$ \\
					\hline
					2 & $\theta_2$ & $d_2$ & $a_2$ & $\alpha_2$ \\
					\vdots & & & & \\
					n & $\theta_n$ & $d_n$ & $a_n$ & $\alpha_n$ \\
					\hline
				\end{tabular}}
			\end{table}

			This is the procedure that is using th D-H parameters
			\begin{align}\label{key}
				A^i_{i-1}(q_i) = Rot(z,\;\theta_i) \; Tran(z,\; d_i)\;  Tran(x,\; a_i)\; Rot(x, \alpha_i)
			\end{align}
		\end{minipage}
		\item Describe full kinematic:
		\begin{equation}\label{key}
			A^n_{0}(q) = A^1_{0}(q1)\cdot A^2_{1}(q2) \cdots  A^n_{n-1}(q_n)
		\end{equation}


	\end{enumerate}
	note-1 with pictures

	\section{Planar double pendulum}

	Simple but not trivial example of a system:

	notes-2
	\begin{table}[H]
		\centering
		\resizebox{8cm}{!}{
		\begin{tabular}{|l|c|c|c|c|}
			\hline
			& $\theta_i$ & $d_i$ & $a_i$ & $\alpha_i$\\
			\hline\hline
			1 & $q_1$ & $0$ & $ a_1$ & $0$ \\
			\hline
			2 & $q_2$ & $0$ & $a_2$ & $0$ \\
			\hline
		\end{tabular}}
	\end{table}

	\begin{align}\label{key}
		q &= \begin{bmatrix}
			q1 \\
			q2
		\end{bmatrix}\\
		A^2_0(q) &= A^1_{0}(q_1) \cdot A^2_{1}(q_2) \\
		Rot(z,q_1) \cdot Tran(x, a_1) &=
		\left[
		\renewcommand\arraystretch{2}
		\begin{array}{c;{2pt/2pt}c}
			\renewcommand\arraystretch{1}
			\begin{matrix}
				c_1 & -s_1 & 0 \\
				s_1 & c_1 & 0 \\
				0 & 0 & 1
			\end{matrix} & \text{\Large 0} \\
		\hdashline[2pt/2pt]
			\text{\Large 0} & \text{\Large 1}
		\end{array}
		\right]=
	\end{align}



 \end{document}
